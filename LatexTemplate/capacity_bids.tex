\documentclass[12pt]{article}

\usepackage{amsmath}

\begin{document}

\section{Mixed-Integer Programming Formulation}

Quick and dirty definitions of the variables and indices.

\begin{tabular}{ll}
$s \in S$ & the set of all scenarios, \\
$j \in J$ & the set of all generators we control, \\
$j \in J^c$ & the set of all generators of the competitors, \\
$i \in I$ & the set of all possible bid prices.
\end{tabular}

The model parameters:

\begin{tabular}{ll}
$p_s$ & the probability that the scenario $s$ will be realised, \\
$d_s$ & the total demand in the scenario $s$, \\
$\lambda^c_{s,j}$ & the price bid by the competitors in the scenario $s$ for their generator $j$, \\
$\bar{g}^c_{s,j}$ & the quantity bid by the competitors in the scenario $s$ for the generator $j$, \\
$g^{c*}_j$ & the maximum quantity which can be produced by the competitors generator $j$, \\
$g^*_j$ & the maximum quantity which can be produced by our generator $j$, \\
$c_j$ & the operating cost to produce one unit of energy by our generator $j$.
\end{tabular}

The model variables:

\begin{tabular}{ll}
$R_s$ & our profit during scenario $s$, \\
$R$ & our average profit for all scenarios, \\
$\pi_s$ & the spot price during scenario $s$, \\
$\lambda_{j}$ & the price bid by us for our generator $j$, \\
$\bar{g}_{j}$ & the quantity bid by us for our generator $j$, \\
$g^c_{s,j}$ & the actual quantity produced by the competitors generator $j$ during scenario $s$, \\
$g_{s,j}$ & the actual quantity produced by our generator $j$ during scenario $s$.
\end{tabular}

The goal is to maximize profit, which can be modeled as such:
\begin{equation}
\max R, \\
\end{equation}
\begin{align}
\mbox{s.t. } \qquad R &= \sum_{s \in S} p_s R_s, \\
R_s &= \sum_{j \in J} \left(\pi_s - c_j\right) g_{s,j} & &\forall s \in S, \\
\sum_{j \in J} g_{s,j} + \sum_{j \in J^C} g^c_{s,j} &= d_s & &\forall s \in S, \\
g^c_{s,j} &\ge 0 & &\forall s \in S, j \in J^c, \\
g^c_{s,j} &\le \bar{g}^{c}_j & &\forall s \in S, j \in J^c, \\
\lambda^c_{s,j} < \pi_s &\rightarrow g^c_{s,j} = \bar{g}^{c}_j & &\forall s \in S, j \in J^c, \\
\lambda^c_{s,j} > \pi_s &\rightarrow g^c_{s,j} = 0 & &\forall s \in S, j \in J^c, \\
\bar{g}_{s,j} &\ge 0 & &\forall s \in S, j \in J, \\
\bar{g}_{s,j} &\le g^*_j & &\forall s \in S, j \in J, \\
g_{s,j} &\ge 0 & &\forall s \in S, j \in J, \\
g_{s,j} &\le \bar{g}_j & &\forall s \in S, j \in J, \\
\lambda_{j} < \pi_s &\rightarrow g_{s,j} = \bar{g}_j & &\forall s \in S, j \in J, \\
\lambda_{j} > \pi_s &\rightarrow g_{s,j} = 0 & &\forall s \in S, j \in J, \\
\pi_s &\in \left\{ \lambda^c_{s,j}, \forall j \in J^c \right\} \cup \left\{ \lambda_j, \forall j \in J \right\} & &\forall s \in S.
\end{align}


It should be noted that in this formulation, the spot price $\pi_s$ is not
set to the lowest price for which all the bids of lower value satisfy the
demand, but instead is set to the highest price for which all the bids of
lower value does not surpass the demand. While this is different from the
real definition of the spot price, we could reduce our own bid quantities
by some infinitesimal amount and get the same effect, with only an
infinitesimal reduction in profit.

Since that formulation is highly inconvenient, we would like to switch to
a linear or bilinear formulation. In order to do so, we need to simplify
the constraints on the spot prices $\pi_s$. If the bid prices $\lambda_j$ were
constrained to be from a finite subset of values, then $\pi_s$ could be
easily defined using those same values.

In this bidding problem, we can prove that the optimal value for each bid
price $\lambda_j$ is always equal to one of the possible competitor bid
prices $\lambda^c_{s,j}$, assuming that ties in prices are resolved in our
advantage.

\textbf{Proof:} Let assume that the one of our bids $\lambda_j$ is strictly
between two other bids (either from ourself or from competitors)
$\lambda^c_{s,j_1}$ and $\lambda^c_{s,j_2}$ (with no other bids in between):
\begin{equation}
\lambda^c_{s,j_1} < \lambda_j < \lambda^c_{s,j_2},
\end{equation}
then the expected profit $R$ would either increase or remain constant if
$\lambda_j$ was to be increased to $\lambda^c_{s,j_2}$, for all possible
scenarios. The possibilities are:
\begin{itemize}
\item $\lambda^c_{s,j_2} \le \pi_s$: The bid price is above $\lambda^c_{s,j_2}$,
so any value of $\lambda_j$ lesser or equal than $\lambda^c_{s,j_2}$ would
not change the
bid price. And since it implies that $\lambda_j \le \pi_s$, then all of the
production for generator $j$ will be sold, for any bid value.
So $R_s$ would stay constant from this change in bid price.
\item $\lambda^c_{s,j_1} \ge \pi_s$: The bid price is below $\lambda_j$, so
none of the production from generator $j$ is sold. Increasing the bid price
does not change that fact, so $R_s$ would remain constant from this change
in bid price.
\item $\lambda_j = \pi_s$: The bid price is equal to $\lambda_j$. This means
that the production from generator $j$ is necessary to satisfy the scenario
demand. Therefore, increasing the bid price will also increase the spot
price by the same amount, as long as the bid price is not increased past
another bid price ($\lambda^c_{s,j_2}$ in this case). So, increasing
$\lambda_j$ up to $\lambda^c_{s,j_2}$ would increase $R_s$.
\end{itemize}
Since increasing $\lambda_j$ to the next higher bid $\lambda^c_{s,j_2}$
always lead to an increasing or a constant $R_s$, the same can be said of
the average profits $R$. It is thus always optimal to only select bid prices
amongst the set of all possible competitors bid prices.

We will denote using $\lambda^c_i$ a competitor bid price from any scenario,
using the $i$ index instead of $s,j$. Those prices are assumed to be
sorted in ascending order ($i' > i \rightarrow \lambda^c_{i'} \ge \lambda^c_i$).
This adds extra variables to our model:

\begin{tabular}{ll}
$y_{s,i}$ & binary variable that indicates if the spot price in scenario $s$ is set to $\lambda^c_i$, \\
$x_{ij}$ & binary variable that indicates if $\lambda_j$ is set to $\lambda^c_i$.
\end{tabular}

Rewriting our model, we now have:

\begin{equation}
\max R, \\
\end{equation}
\begin{align}
\mbox{s.t. } \qquad R &= \sum_{s \in S} p_s R_s, \\
R_s &= \sum_{j \in J} \left(\sum_{i \in I} y_{s,i} \lambda^c_i - c_j\right) g_{s,j} & &\forall s \in S, \label{eqn:Rs} \\
\bar{g}_j &\ge 0 & &\forall s \in S, j \in J, \\
\bar{g}_j &\le g^*_j & &\forall s \in S, j \in J, \\
x_{ij} &\in \{0,1\} & &\forall i \in I, j \in J, \\
\sum_{i \in I} x_{ij} &= 1 & &\forall j \in J, \\
y_{s,i} &\in \{0,1\} & &\forall s \in S, i \in I, \\
\sum_{i \in I} y_{s,i} &= 1 & &\forall s \in S, \\
g_{s,j} &\ge 0 & &\forall s \in S, j \in J, \\
g_{s,j} &\ge \bar{g}_j x_{ij} y_{s,i'} & &\forall s \in S, j \in J, i, i' \in I, i < i' \label{eqn:big_gsj_ge} \\
g_{s,j} &\le \bar{g}_j & &\forall s \in S, j \in J, \\
g_{s,j} &\le \bar{g}_j \left( 1 - x_{ij} y_{s,i'} \right) & &\forall s \in S, j \in J, i, i' \in I, i > i', \label{eqn:big_gsj_le} \\
\sum_{j \in J} g_{s,j} + \sum_{\substack{j \in J^c \\ \lambda^c_{s,j} < \lambda_i}} \bar{g}^c_{s,j} &\le d^s y_{s,i} + \left(\max_{s' \in S} d^{s'} \right) (1 - y_{s,i}) & &\forall s \in S, i \in I, \\
\sum_{j \in J} g_{s,j} + \sum_{\substack{j \in J^c \\ \lambda^c_{s,j} \le \lambda_i}} \bar{g}^c_{s,j} &\ge d^s y_{s,i} & &\forall s \in S, i \in I.
\end{align}

Since some of those constraints have non-linear terms, we need to linearize
them. Constraint (\ref{eqn:big_gsj_ge}) can be replaced by:
\begin{equation}
g_{s,j} \ge \bar{g}_j - g^*_j (2 - x_{ij} - y_{s,i'}) \qquad \forall s \in S, j \in J, i, i' \in I, i < i',
\end{equation}
constraint (\ref{eqn:big_gsj_le}) can be replaced by:
\begin{equation}
g_{s,j} \le g^*_j (2 - x_{ij} - y_{s,i'}) \qquad \forall s \in S, j \in J, i, i' \in I, i > i',
\end{equation}
while each $y_{s,i} g_{s,j}$ term in equation (\ref{eqn:Rs})
must be replaced by a new variable $yg_{s,ij}$ which follows the following
constraints:
\begin{align}
yg_{s,ij} &\ge 0, \\
yg_{s,ij} &\le g_{s,j}, \\
yg_{s,ij} &\le g^*_j y_{s,i}, \\
yg_{s,ij} &\ge g_{s,j} - g^*_j (1 - y_{s,i}).
\end{align}

With this formulation, we have a total of $O\left( |S| |J| |I| \right)$
continuous variables, $O\left( (|S| + |J|) |I| \right)$ binary variables, and
$O\left( |S| |J| |I|^2 \right)$ constraints.



\section{Shortest-Path Algorithm}

We now relax the strategic-pricing problem to allow us to split the capacity
of each generator into an arbitrary number of unique bids. The generators
that will be used to fulfill the bids will then simply be the $n$-cheapest
generators, where $n$ is chosen such that their capacity is at least equal
to the amount we end up having to produce. Accordingly, it is no longer
useful to associate each bid to a specific generator. Instead, each bid
will only be associated to a specific price $\lambda_i$. The quantity bid
on price $\lambda_i$ will be denoted using $g_i$.

Before introducing the shortest-path algorithm to solve the relaxed
strategic-pricing problem, we have to prove a few properties.
We define the cumulative bid quantities $G_i = \sum_{\substack{i' \in I\\i'
\le i}} g_{i'}$, the residual demands $r_{s,i} = d_s - \sum_{\substack{j \in
J^c\\ \lambda^c_{s,j} < \lambda_i}} \bar{g}^c_{s,j}$, and the effective capacities
$c^e_i = \sum_{\substack{j \in J\\c_j < \lambda_i}} g^*_j$.
We also note that we can alternatively define the bids using the
cumulative bid quantities $G_i$ instead of the bid quantities $g_i$, with
the restrictions that $G_{i'} \ge G_i$ if $i'>i$ (do not allow negative bid
quantities) and $G_i \le c^e_i$ (do not offer production at a loss).

The first property we need is that impact on profit of modifying a consecutive
list of cumulative bid quantities $\left\{G_i, \ldots, G_{i'}\right\}$
only depends on the modified quantities and their two neighbouring quantities
$G_{i-1}$ and $G_{i'+1}$.

\textbf{Proof:} Let consider a specific scenario $s$. When calculating the
profits $R_s$, there are three possibilities:
\begin{itemize}
\item $G_{i-1} > r_{s,i}$: the bids at values lower than $\lambda_i$ are
sufficient to satisfy the whole demand. Therefore, the spot price $\pi_s <
\lambda_i$ and will not vary based on $\left\{G_i, \ldots, G_{i'}\right\}$.
The production sold will also not be depending on the modified bids. Therefore,
$R_s$ will be constant.
\item $G_{i'+1} \le r_{s,i'+1}$: the bids at values lower than $\lambda_{i'+1}$
are not sufficient to satisfy the whole demand. Therefore, the spot price
$\pi_s \ge \lambda_{i'+1}$ and will not vary based on
$\left\{G_i, \ldots, G_{i'}\right\}$. The production sold will also not be
depending on the modified bids, since all bids up to $G_{i'+1}$ will be
fully sold. Therefore, $R_s$ will be constant.
\item $G_{i-1} \le r_{s,i} \wedge G_{i'+1} > r_{s,i'+1}$: the bids at lower
values than $\lambda_i$ does not suffice to satisfy the whole demand, and
the bids at lower values than $\lambda_{i'}$ will be enough. In this
case, the precise values of the modified bid quantities
$\left\{G_i, \ldots, G_{i'}\right\}$ will have an impact on the scenario
profit. Because of the values of $G_{i-1}$ and $G_{i'+1}$, the spot price
$\pi_s$ must be in $\left\{\lambda_{i}, \ldots, \lambda_{i'}\right\}$.
Which one will be the spot price will depend on the precise values of the
cumulative bid quantities. Since the profits for a single scenario
only depend on the spot price and the cumulative bid quantity at that
value, then they will only depend on $\left\{G_i, \ldots, G_{i'}\right\}$.
\end{itemize}
Since for each scenario the profit is only a function of the modified
cumulative bid quantities or their two neighbouring quantities, the same
will be true for the full profit function $R$, thus proving the property.

A consequence of this property is that the profit function $R$ can be
written as a sum of incremental profits for individual bids:
\begin{equation}
R\left( (\lambda_1, g_1), \ldots, (\lambda_n, g_n) \right) =
R\left( (\lambda_1, g_1) \right) +
\sum_{k=2}^n \left[ R\left( (\lambda_{k-1}, G_{k-1}), (\lambda_k, g_k) \right) -
R\left( (\lambda_{k-1}, G_{k-1}) \right) \right].
\end{equation}

The second property is that at least one of the set of optimal bids has
all of the cumulative bid quantities $G_i$ equal to one of the following
thresholds:
\begin{itemize}
\item The effective capacity at that price, $c^e_i$.
\item The residual demand for some scenario $s$ and for the next higher
bid price $r_{s,i+1}$, if lower than the effective capacity.
\item The bid quantity is equal to the following bid quantity $G_{i+1}$,
which itself must be at a threshold. Recursively, we can find that these
thresholds are all of the $r_{s,i'}$ for $i' > i+1$.
\end{itemize}

\textbf{Proof:} This property can be proven by showing that if we have
a solution, then the solution can always be improved by moving one of
the $G_i$, as long as all the $G_i$ are not at thresholds.
There are two cases to consider: $G_i > c^e_i$ and $G_i < c^e_i$.

In the case where $G_i > c^e_i$, then it means that we are offering to
sell power at a value lower than our marginal cost. Therefore, it is
trivial to see that reducing $G_i$ to $c^e_i$ will always increase
our profit (if that power was to be sold at that value), or leave it
unchanged (if the bid price happened to be different).

So only the $G_i < c^e_i$ case remains.
Let again consider a specific scenario $s$, and a set of
bids such that for some $i$, $G_i$ is not at any of the preceding thresholds,
and $G_i < G_{i+1}$. It will always be possible to find such a $i$ if at
least single $G_{i'}$ is not at a threshold, since if $G_i$ is not at a
threshold, then either $G_i < G_{i+1}$, or $G_{i+1}$ is also not at a
threshold, since by construction of the thresholds. In that case, increasing
$G_i$ to its next threshold value will depend on the value of the spot price
$\pi_s$:
\begin{itemize}
\item $\pi_s < \lambda_i$: the demand is fully covered by previous bids, so
how is bid at $\lambda_i$ does not matter. So increasing $G_i$ leaves
$R_s$ unchanged.
\item $\pi_s > \lambda_{i+1}$: the demand requires at least the bids up to
values $\lambda_{i+2}$ to reach the demand. This indicates that bidding up
to $G_{i+1}$ will not be sufficient to reach the demand. Since $G_i$ cannot
be increased past $G_{i+1}$ and $G_{i+1}$ is not modified,
then the spot price will also not be modified. So the $R_s$ will again remains
unchanged.
\item $\pi_s = \lambda_i$: the demand is only satisfied using the bids at
value $\lambda_i$. If $G_i < r_{s,i}$, increasing $G_i$ increases how much
power we sell at that bid price, therefore also increasing profit
(since $\lambda_i$ is higher than the marginal production cost).
If $G_i \ge r_{s,i}$, then we are already selling as much as the demand
allows, and increasing $G_i$ does not change the amount of power sold, thus
leaving $R_s$ unchanged. In both cases the spot price stays at $\lambda_i$.
\item $\pi_s = \lambda_{i+1}$: This is the only situation where increasing
the value of $G_i$ could lead to a decrease in the spot price value. For
the spot price to decrease from $\pi_s = \lambda_{i+1}$ to
$\pi_s = \lambda_i$, then the cumulative quantity bid by us at this price
level must be higher than the residual demand at the next next level:
$G_i > r_{s,i+1}$. But since $r_{s,i+1}$ is one of the thresholds, we are
forbidden from increasing $G_i$ past its value, so the spot price will remain
constant. Therefore, increasing $G_i$ will not have any impact on the profit,
since the profit only depends on the quantity bid at the spot price level
$G_{i+1}$, which is not modified here.
\end{itemize}
Since increasing a $G_i$ that is lower than $c^e_i$ and $G_{i+1}$ and not
at any of its threshold values always either increases $R_s$ or leaves it
constant, the same can be said for the full profit function $R$.

This second property indicates that an optimal set of bids can always be
found by only looking at bids where the $G_i$, and thus also the $g_i$,
take specific values (the $c^e_i$ or $r_{s,i'}$ thresholds). Since there is
only a finite number of those values, an algorithm that would checked
all combinations would find the optimal solution in finite time. But such
an algorithm would be very slow, taking a time exponential in the size
of the problem to find the solution. This is where the first property can
be used: since the profit function $R$ can be written as an incremental
sum over individual bids, we can optimise each bids iteratively.
Defining $R^{max}_i(G_i)$ as the maximum profit when bidding for a total
quantity $G_i$ at price up to $\lambda_i$, we have:
\begin{equation}
R^{max}_i(G_i) = \max_{\substack{G_{i-1} \in \left\{ c^e_{i-1}, G_i \right\}
\cup \left\{ r_{s,i} \forall s \in S \right\}
\\ 0 \le G_{i-1} \le G_i}}
R^{max}_{i-1}(G_{i-1}) +
\left[  R\left( (\lambda_{i-1}, G_{i-1}), (\lambda_i, G_i - G_{i-1}) \right) -
R\left( (\lambda_{i-1}, G_{i-1}) \right) \right].
\end{equation}
The optimal profit is then readily obtained by taking the value of
$R^{max}_i(c^e_i)$ associated with the highest bid value $\lambda_i$ of
the problem.

Since there are $O(|S| |J^c|)$ (where $|J^c|$ is used as a shorthand for
the average number of competitor bids per scenario) different bid prices,
and also $O(|S| |J^c|)$ different values for the bid quantities $G_i$,
this mean that finding the optimal set of bids requires the
computation of $O(|S|^3 |J^c|^2)$ profit increments.

The single-scenario single-bid profit function $R_s((\lambda_i, G_i))$
has three regimes: none of the bid is sold, a partial amount of is sold
at the value given, or all of it is sold at a potentially higher spot price.
\begin{equation}
R_s((\lambda_i, G_i)) = \left\{
	\begin{array}{cl}
	0 & \mbox{if } r_{s,i} < 0, \\
	\lambda_i r_{s,i} - c(r_{s,i}) & \mbox{if } 0 \le r_{s,i} < G_i, \\
	G_i \left( \min_{i' \ge i} \left\{ \lambda_{i'} : r_{s,i'+1} < G_i \right\} \right) - c(G_i) & \mbox{if } G_i \le r_{s,i}.
	\end{array}
\right.
\end{equation}
Where $c(G)$ is the total cost to produce $G$ units of energy with the
cheapest available generator. This function can be pre-calculated for all
possible thresholds in advance in $O(|S| |J^c| |J| + |J| \lg |J|)$ time.
These profit functions can be pre-calculated in $O(|S|^2 |J^c|^2)$ time,
and used to calculate the profit increments.
\begin{equation}
R_s\left( (\lambda_{i-1}, G_{i-1}), (\lambda_i, G_i - G_{i-1}) \right) -
R_s((\lambda_i, G_i)) = \left\{
	\begin{array}{cl}
	0 & \mbox{if } r_{s,i} < G_{i-1}, \\
	R_s((\lambda_i, G_i)) - R_s((\lambda_{i-1}, G_{i-1})) & \mbox{if } G_{i-1} \le r_{s,i}.
	\end{array}
\right.
\end{equation}
This last formulation has a simple interpretation: there are two possibilities
when adding an additional bid to a previous bid. Either there is no left-over
demand at the new bid price ($r_{s,i} < G_{i-1}$) and the system is unchanged
by the additional bid; or there is some left-over demand at the new bid price
($G_{i-1} \le r_{s,i}$), and then the spot price will at least be $\lambda_i$,
and the profits will as if there was only a single bid at that value.
Since the full profit function is the sum over all scenarios of the
single-scenario profit functions, the time to compute a single profit
increment is $O(|S|)$, which lead to a total running time of the algorithm
of $O(|S|^4 |J^c|^2)$.

\end{document}

