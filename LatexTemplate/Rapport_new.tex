%\documentclass[letterpaper]{article}
%\usepackage{natbib,alifexi}
\documentclass[a4paper,11pt]{article}
\usepackage[T1]{fontenc}
\usepackage[utf8]{inputenc}
\usepackage{lmodern}

\title{A Mixed-Integer Programming Formulation Applied for Optimal Bids in an Electricity Spot Market}
\author{Gérard Tio Nogueras \\
\mbox{}\\
Université Libre de Bruxelles, Bruxelles, Belgique \\
gtionogu@ulb.ac.be}


\begin{document}
\maketitle
\newpage

\begin{abstract}
\end{abstract}

\newpage

\section{Introduction}
\textit{General Explanation of the main problem and introduce the different approaches.
Explaining the principal mathematics used in the paper to introduce the following sections properly.}

\section{Presentation of the different approaches}
\textit{Present Fampa's approaches using heuristics and MILP and trying to summarize Main paperwork.}

\section{New Formulation of the problem Fampa}
\textit{Explain the main difference}
\subsection{Proposition - optimal bid price}
\textit{Short text introducing the proposition} 
\subsection{Proof}
\textit{Introduction to the proof with the comparison of our price bid to the precedent and following one. Then explaining why the profit either increases or stays equal by comparing all the possibilities. Good idea to introduce graphics for better understanding ? Explaining the use of the two new binary variables to our model and rewrite the model with those variables. Showing that we face a problem with 2 of the inequalities(they are non-linear) and the solution(proof the solution with a matrix cf réunion Etienne). Finally give a quick conclusion to this formulation.}

\section{Shortest Path Algorithm}
\textit{Explain how it's a relaxation of the problem, introduce new variables needed}
\subsection{1st proposition - impact on profit}
\textit{Short text introducing the proposition with the visuals to help understand }
\subsection{Proof}
\textit{For a given scenario, let's compare the of the 3 different possibilities. End with a short conclusion with the huge R function. Important Add graphics cf réunion Etienne}
\subsection{2nd proposition - Thresholds}
\textit{Short text introducing the proposition and explaining the different thresholds.}
\subsection{Proof}
We start by explaining the thesis of the proof. We compare the 2 different cases where $G_i$, the cumulative bid quantities, are not at thresholds.
For the second case we will discuss the dependency of increasing $G_i$ to the next threshold with the spot price. Good idea to show the comparisons with graphs for better understanding. End with a conclusion.

\subsection{The shortest path}
\textit{We will use the 2 precedent propositions to explain $R_max$ solution. Explain the single bid profit function and its 3 regimes. End with a huge conclusion on the algorithm and its use. Use graphics to explain how this model can be seen as a shortest path and to explain the logic behind $R_max$}



\end{document}
